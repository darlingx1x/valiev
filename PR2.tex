\documentclass[a4paper,12pt]{article}

% --- Подключение пакетов ---
\usepackage[utf8]{inputenc}
\usepackage[T2A]{fontenc}
\usepackage[russian]{babel}
\usepackage{graphicx}         % Пакет для картинок
\usepackage{geometry}
\geometry{top=2cm, bottom=2cm, left=3cm, right=1.5cm}
\usepackage{hyperref} 
\usepackage{titlesec}

% --- Данные для титульного листа ---
\title{Групповой отчет на тему: \\ 
\textbf{«Сравнение эффективности создания презентаций в PowerPoint и LaTeX»}}

\author{
  \textbf{Группа разработчиков (067-22):} \\
  1. Ли Никита (Лидер группы) \\
  2. Иванов Иван \\
  3. Петров Петр \\
  4. Сидоров Сидор
}
\date{\today}

\begin{document}

% Титульный лист
\maketitle
\thispagestyle{empty}
\newpage

% Оглавление
\tableofcontents
\newpage

% --- Введение ---
\section{Введение}
Целью данного отчета является сравнительный анализ двух популярных инструментов для создания презентаций: Microsoft PowerPoint и пакета Beamer в системе LaTeX. Актуальность темы подтверждается исследованиями Иванова и Петрова \cite{compare_article}.

% --- Глава 1 ---
\section{Анализ Microsoft PowerPoint}

\subsection{Преимущества}
PowerPoint является стандартом в бизнес-среде \cite{ppt_manual}.
\begin{itemize}
    \item Интуитивно понятный интерфейс (WYSIWYG).
    \item Легкость вставки мультимедиа.
    \item Низкий порог входа.
\end{itemize}

\subsection{Недостатки}
Однако, при работе с большими документами возникают проблемы:
\begin{itemize}
    \item Сложности с версткой математических формул.
    \item «Поплывшее» форматирование при открытии на других версиях ПО.
\end{itemize}

% --- Глава 2 ---
\section{Анализ LaTeX (Beamer)}

\subsection{Преимущества подхода Code-based}
Как отмечает Дональд Кнут \cite{knuth}, системы верстки обеспечивают идеальное качество печати и отображения.
\begin{enumerate}
    \item \textbf{Стабильность:} Презентация выглядит одинаково на любом компьютере.
    \item \textbf{Математика:} Лучший инструмент для формул: $E=mc^2$.
    \item \textbf{Автоматизация:} Оглавление и списки строятся автоматически.
\end{enumerate}

\subsection{Сложности использования}
Главный минус — необходимость знания синтаксиса кода, что подробно описано в документации \cite{overleaf_doc}.

% --- Сравнительная таблица ---
\section{Сравнение характеристик}
В таблице \ref{tab:comparison} представлено итоговое сравнение инструментов.

\begin{table}[h]
    \centering
    \begin{tabular}{|l|c|c|}
        \hline
        \textbf{Критерий} & \textbf{PowerPoint} & \textbf{LaTeX (Beamer)} \\
        \hline
        Скорость создания (простая) & Высокая & Низкая \\
        \hline
        Качество формул & Среднее & Высокое \\
        \hline
        Совместная работа & Online Office & Overleaf \\
        \hline
        Цена & Платный (часто) & Бесплатный \\
        \hline
    \end{tabular}
    \caption{Сравнительная таблица PowerPoint и LaTeX}
    \label{tab:comparison}
\end{table}

% --- Графическая иллюстрация ---
\section{Визуальное сравнение}
На рисунке \ref{fig:logos} показан процесс работы в разных средах.

\begin{figure}[h]
    \centering
    % ИСПРАВЛЕНО: добавлено явное расширение .png
    % Убедись, что твой файл в проекте называется ppt_latex.png
    \includegraphics[width=0.7\textwidth]{ppt_latex.png}
    \caption{Логотипы инструментов}
    \label{fig:logos}
\end{figure}

% --- Совместная работа ---
\section{Организация совместной работы}
Работа над отчетом велась в системе Overleaf. Группа из 4 человек (см. список авторов) использовала функцию «Track Changes» для отслеживания версий, что позволило избежать конфликтов редактирования.

% --- Заключение ---
\section{Заключение}
Выбор инструмента зависит от задачи. Для быстрых визуальных докладов лучше подходит PowerPoint. Для научных докладов с большим количеством формул идеальным выбором является LaTeX.

% --- БИБЛИОГРАФИЯ (BIBTEX) ---
\newpage
% Стиль оформления (plain - простой нумерованный)
\bibliographystyle{plain} 
% Подключение файла references.bib
\bibliography{references}
\addcontentsline{toc}{section}{Список литературы}

\end{document}