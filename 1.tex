\documentclass[a4paper,12pt]{article}

% --- Подключение пакетов ---
\usepackage[utf8]{inputenc}   % Кодировка
\usepackage[T2A]{fontenc}     % Шрифты
\usepackage[russian]{babel}   % Русский язык
\usepackage{graphicx}         % Для вставки картинок
\usepackage{amsmath, amssymb} % Для математических формул
\usepackage{geometry}         % Настройка полей
\geometry{top=2cm, bottom=2cm, left=3cm, right=1.5cm}
\usepackage[hidelinks]{hyperref} % Для кликабельного оглавления и ссылок
\usepackage[normalem]{ulem}   % Для подчеркивания текста




\title{Отчет по практической работе №1 \\ \large Решение задач 2 и 3 }
\author{Студент: ЛИ НИКИТА \\ Группа: 067-22} 
\date{\today}

\begin{document}


\maketitle
\thispagestyle{empty}
\newpage


\tableofcontents
\newpage


\section*{Введение}
\addcontentsline{toc}{section}{Введение}

В современном мире информация стала одним из самых ценных ресурсов. \textbf{Кибербезопасность} — это набор методов и технологий, предназначенных для защиты компьютеров, сетей, программ и данных от несанкционированного доступа. Актуальность темы обусловлена стремительным ростом цифровизации общества. Целью данной работы является анализ основных угроз и методов защиты информации, а также освоение базовых принципов верстки документов в системе \textit{LaTeX}.

% ================= ГЛАВА 1: ЗАДАНИЕ 1 =================
\section{Задание 1. Кибербезопасность}

\subsection{Виды угроз информационной безопасности}
Вредоносное ПО (Malware) остается одной из главных угроз. К нему относятся вирусы, черви, трояны и программы-вымогатели. Основная цель таких программ — \textit{нарушение конфиденциальности, целостности или доступности} информации. Атаки на «человеческий фактор» часто эффективнее технических взломов. \uline{Фишинг} — это вид интернет-мошенничества, целью которого является получение доступа к конфиденциальным данным пользователей — логинам и паролям. Атаки типа «отказ в обслуживании» (DDoS) направлены на то, чтобы сделать ресурс недоступным для легитимных пользователей. Для этого злоумышленники используют сети зараженных устройств, называемые \textbf{ботнетами}.

\subsection{Технические средства защиты}
Шифрование данных является фундаментом безопасности. Оно преобразует информацию в нечитаемый вид для всех, кто не обладает специальным ключом дешифрования. Используются симметричные и \textit{асимметричные алгоритмы}. Для защиты периметра сети используются межсетевые экраны (Firewalls). Они фильтруют трафик на основе заданных правил. Системы обнаружения вторжений (IDS) анализируют сеть на предмет \uline{аномальной активности}. Парольной защиты уже недостаточно. Современные стандарты требуют внедрения \textbf{многофакторной аутентификации (MFA)}, где помимо пароля требуется второй фактор (например, код из SMS или биометрия).

\subsection{Организационно-правовое обеспечение}
В Российской Федерации основным законом является ФЗ-152 «О персональных данных». Он регулирует деятельность операторов, осуществляющих обработку данных, и накладывает на них \textit{строгие обязательства}. Каждая компания должна иметь утвержденную политику безопасности. Этот документ регламентирует правила работы сотрудников с информацией, использование съемных носителей и порядок действий при инцидентах. Согласно статистике, более 80\% инцидентов происходят по вине сотрудников. Регулярные тренинги по \uline{цифровой гигиене} позволяют существенно снизить риски успешных фишинговых атак.

\newpage

% ================= ГЛАВА 2: ЗАДАНИЕ 2 =================
\section{Задание 2. Эволюция компьютерных интерфейсов}

Пользовательский интерфейс (UI) — это средство, с помощью которого человек взаимодействует с машиной. Рассмотрим три основных этапа его развития.

\subsection{Этап 1: Текстовый интерфейс (CLI)}
Изначально общение с компьютером происходило через командную строку (Command Line Interface). Пользователь вводил текстовые команды с клавиатуры. Это требовало запоминания сложного синтаксиса и понимания структуры команд (см. рис.~\ref{fig:cli}).


\begin{figure}[h]
    \centering
    \includegraphics[width=0.7\textwidth]{cli} 
    \caption{Пример текстового интерфейса (MS-DOS)}
    \label{fig:cli}
\end{figure}

\subsection{Этап 2: Графический интерфейс (GUI)}
С появлением первых графических интерфейсов (например, Xerox Alto, Apple Macintosh) и развитием операционных систем, таких как Microsoft Windows, произошла революция в пользовательском опыте. Появилась метафора «Рабочего стола», окна, иконки, меню и управление с помощью мыши. Как показано на рис.~\ref{fig:gui}, информация стала гораздо более наглядной и интуитивно понятной.


\begin{figure}[t]
    \centering
    \includegraphics[width=0.7\textwidth]{gui}
    \caption{Классический графический интерфейс Windows 95}
    \label{fig:gui}
\end{figure}

\subsection{Этап 3: Сенсорный интерфейс (NUI)}
Современный этап развития интерфейсов связан с распространением мобильных устройств. Здесь управление осуществляется непосредственно касаниями пальцев по экрану (тачскрин), а также жестами и голосовыми командами. Этот тип интерфейса стремится сделать взаимодействие с устройством максимально интуитивным. Пример представлен на рис.~\ref{fig:touch}.


\begin{figure}[b]
    \centering
    \includegraphics[width=0.4\textwidth]{touch}
    \caption{Современный сенсорный интерфейс смартфона}
    \label{fig:touch}
\end{figure}

\clearpage

% ================= ГЛАВА 3: ЗАДАНИЕ 3 =================
\section{Задание 3. Вёрстка математической формулы}

Согласно варианту №6, необходимо набрать следующее дифференциальное уравнение с использованием команд \LaTeX:

\vspace{1cm}

\begin{equation}
    \frac{\ctg x}{y} y' = 1
\end{equation}

\vspace{1cm}

Код данной формулы: \texttt{\textbackslash frac\{\textbackslash ctg x\}\{y\} y' = 1}.

\newpage

% --- Заключение ---
\section*{Заключение}
\addcontentsline{toc}{section}{Заключение}

В ходе работы были рассмотрены ключевые аспекты кибербезопасности, основные виды угроз и методы защиты информации. Изучены этапы эволюции компьютерных интерфейсов от текстовых до сенсорных. Также выполнена верстка сложной математической формулы. Полученные навыки работы в \LaTeX{} позволяют создавать профессионально оформленные научные документы.

% --- Список литературы ---
\section*{Список литературы}
\addcontentsline{toc}{section}{Список литературы}

\begin{enumerate}
    \item \textit{Бабаш А. В., Баранова Е. К.} Информационная безопасность и защита информации. — М.: РИОР, 2021. — 324 с.
    \item \textit{Гафнер В. В.} Информационная безопасность: учебное пособие. — Ростов н/Д: Феникс, 2020. — 324 с.
    \item \textit{Гридина Е. Г.} Управление защитой информации. — М.: МГТУ им. Баумана, 2019. — 216 с.
    \item \textit{Партыка Т. Л., Попов И. И.} Информационная безопасность. — М.: Форум, 2022. — 432 с.
    \item \textit{Таненбаум Э., Уэзеролл Д.} Компьютерные сети. 5-е изд. — СПб.: Питер, 2020. — 960 с.
    \item \textit{Шаньгин В. Ф.} Информационная безопасность и защита информации. — М.: ДМК Пресс, 2021. — 702 с.
    \item Федеральный закон от 27.07.2006 N 152-ФЗ «О персональных данных» // Собрание законодательства РФ. — 2006.
    \item \textit{Столлингс В.} Криптография и сетевая безопасность: принципы и практика. — М.: Вильямс, 2018. — 800 с.
    \item \textit{Шнайер Б.} Секреты и ложь. Безопасность данных в цифровом мире. — СПб.: Питер, 2019. — 368 с.
    \item \textit{Касперский Е. В.} Компьютерное зловредство. — М.: Питер, 2020. — 208 с.
\end{enumerate}

\end{document}